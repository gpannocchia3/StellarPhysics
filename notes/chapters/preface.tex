\chapter*{Preface}

\begin{quote}
    \emph{In a sense, you might say I'm getting quite fond of writing these kind of notes}.
\end{quote}
Not long ago, it occurred to me how cool it is when someone unexpectedly releases a very detailed and all-comprehensive version of their notes, especially when dealing with a course that has a handful of topics often interacting together in unpredictable, yet fascinating, ways.
\par These notes will be mainly based on \emph{my} own notes of the lectures by Professor Scilla Degl'Innocenti during the academic year 2025-2026. However, since I take little to no pride in my messy notes, I'll be often using some of the references you can find on the course catalogue page or that have been cited during class. Either way, I'll do my best keeping track of them all in the bibliography of this humble collection.
\par You can report errors (whatever their nature might be) and suggestions for additions at g.pannocchia3@studenti.unipi.it or through whatever convoluted way (conventional or not) you prefer\footnote{I'd like, however, not to see my house stormed by homing pidgeons.}.
\par  You can find all the notes I've redacted \href{https://github.com/gpannocchia3}{here}.
\par Without further ado, we'd better not lose much more time on a preface and get started with it.
\begin{quote}
    \emph{There was Eru, the One, who in Arda is called Ilúvatar; and he made first the Ainur} [...] \emph{But for a long while they sang only each alone, or but few together, while the rest hearkened; for each comprehended only that part of the mind of Ilúvatar from which he came, and in the understanding of their brethren they grew but slowly. \\
    Yet ever as they listened they came to deeper understanding, and increased in unison and harmony.}
\end{quote}

\begin{flushright}
    \emph{Ainulindalë}, "The music of the Ainur",\\
    Silmarillion, J. R. R. Tolkien
\end{flushright}